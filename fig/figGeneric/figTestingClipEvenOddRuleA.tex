\documaneclass[b5paper]{standalone}
\usepackage{fontspec}
\usepackage{polyglossia}
\usepackage{circuitikz}
\usepackage{ifthen}   
\usepackage{amsmath}
\usetikzlibrary{calc}
\usetikzlibrary{decorations.pathreplacing}
%
\setmainlanguage{english}
\setotherlanguages{arabic}
\newfontfamily\arabicfont[Scale=1.0,Script=Arabic]{Scheherazade}
\newfontfamily\urdufont[Scale=1.0,Script=Arabic]{XB Tabriz}


\begin{document}
\begin{urdufont}
\def\ringa{(-1,0) circle (2) (-1,0) circle (3)}
\def\ringb{(1,0) circle (2) (1,0) circle (3)}

\begin{tikzpicture}
    % First we fill the intersecting area
    % The \clip command does not allow options, therefore 
    % we have to use a scope to set the even odd rule. 
    \begin{scope}[even odd rule]
        % Define a clipping path. All paths outside ringa will
        % be cut because the even odd rule is set. 
        \clip \ringa;
\draw (-4,0)--(4,0);
        % Fill ringb. Since the even odd rule is set, only the
        % ring will be filled, not the hole in the middle.  
        \fill[fill=orange] \ringb;
    \end{scope}
    % Then we draw the rings
    \draw \ringa;
    \draw \ringb;
\end{tikzpicture}
\end{urdufont}
\end{document}
%---------------------

