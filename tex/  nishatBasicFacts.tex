\باب{بنیادی حقیقتونہ}
په دې باب کښ هغه خبرې راېوځاې کړې دې کومې به چه ټول کتاب کښ بېابېا رازې.امېد دې چه د کتاب لوستلو په وخت به په اصل مضمون باندې غور کول اسان وې.
\حصہ{بنیادی اکائی}
په دې کتاب کښ به د غونډې نړې اکائ نظام  استعمالېګې.په دے نظام کښ د تول اکائ کلوګرام، د ناپ اکائ مېټر،او د وخت اکائ سېکنډ دې 

\حصہ{مقداری او سمتیہ}
      
کہ د کراچئ نہ یو الوتکہ دشمال پہ مخ چھ سو ساټھ کلومیټر فی ګھنټہ روان وی نوھعہ بہ پہ دوہ ګھنټوکښ افعانستان کښ مزارشریف تہ اورسی۔پہ دے فقرہ کښ د الوتکےد رفتار مقداراو سمت دواړہ بیان کول ضروری دی۔داسے شے چہ ھعہ مقدار او سمت دواړہ لری، ھعے تہ سمتیہ وئیلی شی پہ دے مثال کښ سمتی رفتار تا سمتیہ دہ۔

دغہ رنګ کہ مونګ د دوہ کلوګرام دغنمو داوړو یا د شپږ لیټرو پټرولو خبرہ اوکو۔نو دے کښ دسمت ھیڅ ذکر نہ رازی ۔ھعہ شے چہ مقدار لری او سمت نہ لری ھعے تہ مقداری وئیلی شی ۔پہ دے مثال کښ وزن او حجم دواړہ مقداری دی۔

پہ دے کتاب کښ بہ مقداری شیزان د انګریزے یا لا طینے ژبے پہ سادہ لکھاے کښ پہ وړو حرفونو کښ یا پہ غټو حرفونو کښ لیکلی کیګی۔او پہ دے کتاب کښ سمتیہ شیزان د انګریزے یا لاطینے ژبے پہ غټہ لکھاے کښ پہ وړو حرفونو کښ یا پہ غټو حرفونو کښ لیکلے کیګی۔مثلا قوت د پارہ بہ ف استعمالیګی۔داسے سمتیہ چہ د ھعے اوږدوالے یو وی ھعے تہ اکائی سمتیہ وئیلی شی۔پہ دے کتاب کښ د انګریزے ژبے وړومبے وړوکے حرف چہ پہ غټہ  لکھاے کښ لیکلی وی اکائی سمتیہ پہ ګوتہ کوی۔ مثلا اکائی سمتیہ  ۱،۲،۳ د خلا درے ګوټونہ پہ ګوتہ کوی۔۱ کښ پہ وړہ لکھائی کښ ۱، دا د خلا ۱،طرف پہ ګوتی کوی۔کہ چرے د سمتیہ اوږدوالے او د ھعے مخ جداجدا لیکل وی نو د ھعے اوږدوالی پہ ګوتہ کولو د پارہ پہ سادہ لکھائی کښ ھعہ حرف استعمالیګی کوم چہ سمتیہ پہ ګوتہ کولو د پارہ پہ غټہ لکھائی کښ استعمال شوی وی۔

 


