\باب{بنیادی حقیقتونہ}
په دې باب کښ هغه خبرې راېوځاې کړې دې کومې به چه ټول کتاب کښ بېابېا رازې.امېد دې چه د کتاب لوستلو په وخت به په اصل مضمون باندې غور کول اسان وې.
\حصہ{بنیادی اکائی}
په دې کتاب کښ به د غونډې نړې اکائ نظام  استعمالېګې.په دے نظام کښ د تول اکائ کلوګرام، د ناپ اکائ مېټر،او د وخت اکائ سېکنډ دې 

\حصہ{مقداری او سمتیہ}
      
کہ د کراچئ نہ یو الوتکہ دشمال پہ مخ چھ سو ساټھ کلومیټر فی ګھنټہ روان وی نوھعہ بہ پہ دوہ ګھنټوکښ افعانستان کښ مزارشریف تہ اورسی۔پہ دے فقرہ کښ د الوتکےد رفتار مقداراو سمت دواړہ بیان کول ضروری دی۔داسے شے چہ ھعہ مقدار او سمت دواړہ لری، ھعے تہ سمتیہ وئیلی شی پہ دے مثال کښ سمتی رفتار تا سمتیہ دہ۔

دغہ رنګ کہ مونګ د دوہ کلوګرام دغنمو داوړو یا د شپږ لیټرو پټرولو خبرہ اوکو۔نو دے کښ دسمت ھیڅ ذکر نہ رازی ۔ھعہ شے چہ مقدار لری او سمت نہ لری ھعے تہ مقداری وئیلی شی ۔پہ دے مثال کښ وزن او حجم دواړہ مقداری دی۔
 
پہ دے کتاب کښ بہ مقداری شیزان د انګریزے یا لا طینے ژبے پہ سادہ لکھاے کښ پہ وړو حرفونو کښ یا پہ غټو حرفونو کښ لیکلی کیګی۔او پہ دے کتاب کښ سمتیہ شیزان د انګریزے یا لاطینے ژبے پہ غټہ لکھاے کښ پہ وړو حرفونو کښ یا پہ غټو حرفونو کښ لیکلے کیګی۔مثلا قوت د پارہ بہ ف استعمالیګی۔داسے سمتیہ چہ د ھعے اوږدوالے یو وی ھعے تہ اکائی سمتیہ وئیلی شی۔پہ دے کتاب کښ د انګریزے ژبے وړومبے وړوکے حرف چہ پہ غټہ  لکھاے کښ لیکلی وی اکائی سمتیہ پہ ګوتہ کوی۔ مثلا اکائی سمتیہ  ۱،۲،۳ د خلا درے ګوټونہ پہ ګوتہ کوی۔۱ کښ پہ وړہ لکھائی کښ ۱، دا د خلا ۱،طرف پہ ګوتی کوی۔کہ چرے د سمتیہ اوږدوالے او د ھعے مخ جداجدا لیکل وی نو د ھعے اوږدوالی پہ ګوتہ کولو د پارہ پہ سادہ لکھائی کښ ھعہ حرف استعمالیګی کوم چہ سمتیہ پہ ګوتہ کولو د پارہ پہ غټہ لکھائی کښ استعمال شوی ۔دا رنګے د سمتیہ ف اوږدوالے بہ ف لیکلے شی۔شکل کښ د سمتیہ ف اوږدوالے ف څلوردے۔کہ چرے د سمتیہ پہ سمت یو اکائی سمتیہ جوړہ کړے شی نو دا اکائی سمتیہ د ھعے سمتیے سمت ظاہروی۔دسمتیہ ف سمت بہ پہ اکائی سمتیہ ا ف لیکلے کیږی۔دلتہ پہ وړوکے لیک کښ  ف دا خبرہ څرګندہ کوی چہ دا اکائی سمتیہ د ف سمت ظاہروی۔پہ شکل کښ ا ف د ا ے برابر دہ ځکہ چہ د ف مخ ښی طرف تہ دے۔     

\حصہ{محدد، خط مرتب}   
   دنیا درے ګوټہ دہ۔پہ دے کښ کہ ہرہ نقطہ واغستے شی نو د ھعے مقام پہ درے محدد ظاہرولے شی۔نورہ دا چہ پہ خلا کښ ہرہ سمتیہ،  یو بل تہ ولاړ د دریو اکائی سمتیو پہ امداد څرہ لیکلے شی۔راځی چہ د محدد یو څو قسمونہ اوګورو۔ 

\جزوحصہ{کارتیسی محدد} 
د خلا یو بل تہ ولاړ، درے اکائی سمتیہ پہ شکل کښ ښودلے شوی دی۔د یو بل تہ ولاړ مطلب دا دے چہ پہ دوی کښ ہر یو اکائی سمتیہ نورو دواړو تہ پہ نوی زاویہ دہ۔دہ دوی  سمت کښ اوږدوالے پہ ا،ب،ګ ظاہرولے شی۔  

   کہ چرے د خی لاس څلور ګوتے د الف د سمت طرف تہ اونیولے شی او بیا دا ګوتے د ب د سمت طرف تہ راتاو کړے شی نو د دے لاس کټہ ګوتہ بہ د ج سمت ظاہری۔دارنګے د خلا، یو بل تہ اولاړ، درے اکائی سمتو نظام د خی لاس نظام بوئی۔

پہ شکل کښ د مرکز نہ تر پ سمتیہ الف ښودلے شوے دہ۔پہ کارتیسی نظام کښ دغہ سمتیہ د دریو سمتیو  پہ مدد څرہ داسے لیکلے کیږی۔

کہ پہ کارتیسی نظام کښ ج صفر کیښودے شی او الف،ب بدلیږی نو مونږ تہ بہ الف ب سطح حاصلیږی۔کہ شکل کښ ف یو نقطہ وی او سطح الف ب مونږ زمکہ اوګنړو نو پہ شکل کښ د ډبی پہ پاسنے سطح د ج قیمت  پہ دریو ټکاو دے یعنی ز=۳خو الف د صفر نہ تر دریو پورے او ب د صفر نہ تر څلورو پورے قیمت لرلے شی۔دغہ رنګے د ډبی پاسنے سطح داسے لیکلے شی۔

کہ چرے د ج قیمت د صفر نہ تر دریو پورے، د الف قیمت د صفر نہ تر دوو پورے او د ب قیمت د صفر نہ تر څلورو پورے بدلیږی نو مونږ تہ بہ پہ شکل کښ د ښودلی ډبی حجم حاصل شی۔دغہ رنګ د دے ډبی حجم بہ  داسے لیکلے شی۔   

\جزوحصہ{نلکی محدد}
 د مرکز نہ  تر نقطہ ف پورے سمتیہ الف پہ شکل کښ ښکاری۔دغہ سمتیہ پہ دوو سمتیو څرہ داسے لیکلے شی۔

 سمتیہ ف پہ الف ب سطح دہ۔د دے شکل نہ ښکارہ دہ چہ





  
