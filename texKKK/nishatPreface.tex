ېو څو کالونه حکومت پاکستان ستر تعلېم باندې توجه ورکې په دې وجه باندې د ملک په تارېخ کښ په وړومبۍ زل په سترو پوهنتونو کښ د تحقېق رجحان پېدا شوې دې.  امېد دې دا سلسله به روانه وې .پاکستان کښ ستر تعلېم په انګرېزې ژبه کښ ښودلې کېږۍ دنېا کښ د تحقېق کار قرېبا په انګرېزې کښ چهاپ کېږې په انګرېزې ژبه کښ په هرموضوع باندې بېشمېره کتابونه شته او د دغې کتابونو نه طالبان اوطالبانې زدهکړه کولې شې.

زدهکړه مورهنې ژبه کښ کېږې.خو افسوس دې چه زمونږ په ملک کښ د پختنو د پاره په پختو کښ هډو کتابونه شته نه.او دا زمونږ د تعلېم نه د محرومه کېدو غټه وجه ده.په پښتنه نړې کښ د پښتنو مشرانولېکوالو په سائنس کښ هېڅ کوشش نه دې کړې.د دې کمې پوره کولودپاره زه دا کتاب لېکم.

 دا کتاب په آسانه پښتو کښ لېکلې شوې دې.دې کتاب کښ د غونډې نړې اکائ نظام استعمال شوې دې.ضرورۍ بدلېدونکې نښې هم هعه اېښې دې کومې چه د ننې نظام تعلېم په کتابونو کښ شته.

امېد دې چه دا کتاب به ېو ورځ په پښتو ژپه کښ د انجنېئرنګ نصابې کتاب د پاره استعمال شې.په پښتو ژپه کښ د الېکټرېکل انجنېئرنګ د غونډ نصاب د پاره دا وړومبې قدم دې

د دې کتاب لوستونکو ته دا خواست دې چه دا کتاب نورو طالبانو ته اورسوې او که چرته په دې کتاب غلطې ولولې نو ما ته د په دې اۍ.مېل باندې خبر اوکړې. 

خالد خان يوسفزئ.








