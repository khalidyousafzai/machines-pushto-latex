\باب{مقناطیسی ادوار}
\حصہ{مزاحمت  اور ہچکچاہٹ}
شکل \حوالہ{شکل_مقناطیسی_دور_مزاحمت_ہچکچاہٹ}  میں ایک سلاخ دکھائی گئی ہے۔ اس کی لمبائی کی سمت میں مزاحمت  یہ ہے
\begin{align}\label{مساوات_مقناطیسی_دور_مزاحمت_کی_تعریف}
R=\frac{l }{\sigma A}
\end{align}
جہاں  \عددیء{\sigma} موصلیت کو ظاہر کرتی ہے اور \عددیء{A=wh}  ہے۔
\begin{figure}
\centering
\includegraphics{figMagneticCircuitsResistanceAndReluctance}
\caption{مزاحمت اور ہچکچاہٹ}
\label{شکل_مقناطیسی_دور_مزاحمت_ہچکچاہٹ}
\end{figure}
اگر اس سلاخ کا مقناطیسی مستقل\فرہنگ{مقناطیسی مستقل}\حاشیہب{permeability, magnetic constant}\فرہنگ{permeability}\فرہنگ{magnetic constant}  \عددیء{\mu} ہو تو اس سلاخ کی ہچکچاہٹ\فرہنگ{ہچکچاہٹ}\فرہنگ{reluctance}\حاشیہب{reluctance} \عددیء{\Re}  یوں بیان کی جائے گی۔
\begin{align}\label{مساوات_مقناطیسی_دور_ہچکچاہٹ_کی_تعریف}
\Re = \frac{l}{\mu A}
\end{align}
مقناطیسی مستقل \عددیء{\mu} کو عموما ً خالی خلاء کی مقناطیسی مستقلکی \عددیء{\mu_0} نسبت سے لکھا جاتا ہے یعنی
\begin{align}
\mu=\mu_r \mu_0
\end{align}
جہاں \عددیء{\mu_r} جزو مقناطیسی مستقل\فرہنگ{مقناطیسی مستقل!جزو}\فرہنگ{relative permeability}\فرہنگ{permeability!relative}  کہلاتی ہے۔ہچکچاہٹ کی اکائی ایمپیئر-چکر فی ویبر  ہے جس کی وضاحت آپ کو جلد ہو جائے گی۔
%
\ابتدا{مثال}
شکل  میں دی گئی سلاخ کی ہچکچاہٹ معلوم کریں
\عددیء{\mu_r=2000 }، \عددیء{l=\SI{10}{\centi \meter}}، \عددیء{h=\SI{3}{\centi \meter}} اور \عددیء{w=\SI{2.5}{\centi \meter}} ہیں۔

حل:
\begin{align*}
\Re& = \frac{l}{\mu_r \mu_0 A}\\
&=\frac{10\times 10^{-2}}{2000 \times 4 \pi \times 10^{-7} \times 2.5 \times 10^{-2} \times 3 \times 10^{-2}}\\
&=\SI{53044}{\ampere \cdot turns \per \weber}
\end{align*}
\انتہا{مثال}

\حصہ{کثافتِ برقی رو  اور برقی میدان کی شدت}\شناخت{حصہ_برقی_دور_کثافت_برقی_رو_اور_میدان}
اگر اس سلاخ کے سروں پر برقی دباؤ \عددیء{v} لاگو کی جائے جیسا کہ شکل  \حوالہ{شکل_مقناطیسی_دور_کثافت_رو_اور_برقی_شدت} میں دکھایا گیا ہے تو اس میں برقی رو \عددیء{i} گزرے گا جس کی مقدار اوہم کے قانون  سے یوں حاصل ہوتی ہے
\begin{align}
i=\frac{v}{R}
\end{align}
اس مساوات کو مساوات \حوالہ{مساوات_مقناطیسی_دور_مزاحمت_کی_تعریف}  کی مدد سے یوں لکھ سکتے ہیں
\begin{align}
i=v \left(\frac{\sigma A}{l}\right)
\end{align}
یا
\begin{align}
\frac{i}{A}=\sigma \left(\frac{v}{l} \right)
\end{align}
اسے مزید یوں لکھ سکتے ہیں
\begin{align}\label{مساوات_مقناطیسی_دور_اوہم_قانون_کی_تفرق_شکل}
J =\sigma E
\end{align}
اگر شکل میں سمتیہ \سمتیہ{J} کا طول \عددیء{J} ہو اور سمتیہ \سمتیہ{E} کا طول \عددی{E} ہو جہاں ان دونوں سمتیہ کی سمت  \عددیء{\ay} ہے تب  اس مساوات کو یوں لکھا جا سکتا ہے۔
\begin{align}
\kvec{J}=\sigma \kvec{E}
\end{align}
یہ دونوں مساوات اوہم کے قانون کی ایک اور شکل ہیں۔ مساوات \حوالہ{مساوات_مقناطیسی_دور_اوہم_قانون_کی_تفرق_شکل}  میں 
\begin{align}
J&=\frac{i}{A} \label{مساوات_مقناطیسی_دور_کثافت_رو}\\
\intertext{اور}
E&=\frac{v}{l} \label{مساوات_مقناطیسی_دور_برقی_شدت}
\end{align}
%
\begin{figure}
\centering
\includegraphics{figMagneticCircuitsCurrentDensityAndElectricFieldIntensity}
\caption{کثافتِ برقی رو اور برقی دباؤ کی شدت}
\label{شکل_مقناطیسی_دور_کثافت_رو_اور_برقی_شدت}
\end{figure}
ہیں۔ شکل سے واضح ہے کہ برقی رو \عددیء{i} سلاخ کی رقبہ عمودی تراش \عددیء{A} سے گزرتی ہے لہٰذا مساوات \حوالہ{مساوات_مقناطیسی_دور_کثافت_رو} کے تحت \عددیء{J} رقی رو کی کثافت کو ظاہر کرتی ہے۔ اسی وجہ سے \عددیء{J} کو کثافتِ برقی رو \فرہنگ{کثافت!برقی رو}\حاشیہب{current density} ہی کہتے ہیں۔ اسی طرح مساوات \حوالہ{مساوات_مقناطیسی_دور_برقی_شدت}   سے یہ واضح ہے کہ \عددیء{E} برقی دباؤ فی اکائی لمبائی کو ظاہر کرتی ہے۔  یوں  \عددیء{E} کو برقی میدان کی شدت\فرہنگ{برقی میدان!شدت}\حاشیہب{electric field intensity} کہتے ہیں۔جہاں متن سے واضح ہو کہ برقی میدان کی بات ہو رہی ہے وہاں اس نام کو چھوٹا کر کے \عددیء{E} کو میدانی شدت  سے پکارا جاتا ہے۔برقی میدان\فرہنگ{برقی میدان}\حاشیہب{electric field}\فرہنگ{electric field} سے مُراد کسی چارج کے اِردگرد وہ جگہ ہے جس میں اس چارج کا اثر محسوس کیا جاتا ہے۔

	ہم بالکل اسی طرح مقناطیسی متغیرہ کے لئے بھی اس طرح کے مساوات لکھ سکتے ہیں۔ حصہ \حوالہ{حصہ_برقی_دور_کثافت_مقناطیسی_بہاو_اور_میدان}  میں بھی یہی کریں گے۔

\حصہ{برقی ادوار}
	برقی دور میں برقی دباؤ\فرہنگ{برقی دباو}\حاشیہب{electric voltage}  \عددیء{V}  کی وجہ سے برقی رو\فرہنگ{برقی رو}\حاشیہب{electric current} \عددیء{i} پیدا ہوتی ہے۔ تانبہ\فرہنگ{تانبہ}\حاشیہب{copper}   کی موصلیت \عددیء{\sigma=\SI{5.9e7}{\siemens \per \meter}} ہے جہاں \عددیء{\si{\siemens \per \meter}} موصلیت کی اکائی ہے۔لہٰذا تانبہ کی بنی تار کی مزاحمت  \عددیء{R_{\textup{تار}}}  قابلِ نظرانداز ہوتی ہے۔اگر ایسی تار میں برقی رو \عددیء{i} کا گزر ہو  تو اس تار کی مزاحمت میں اوہم کے قانون کے تحت  برقی دباؤ  \عددیء{\Delta v=i R_{\textup{تار}}} گھٹے  گی۔\عددیء{R_{\textup{تار}}} کی قابلِ نظر انداز ہونے کی وجہ سے یہ مقدار بھی قابلِ نظر انداز ہی ہو گی۔ اس کا مطلب ہے کہ برقی تار کی مدد سے برقی دباؤ کی ایک جگہ سے دوسری جگہ رسائی بغیر کم ہوئے ممکن ہے۔اسی لئے تانبہ کی تار کو عموما ً برقی دباؤ کی ایک جگہ سے دوسری جگہ رسائی کے لئے استعمال کیا جاتا ہے اور اس کی مزاحمت کو صفر ہی سمجھا جاتا ہے۔ شکل \حوالہ{شکل_مقناطیسی_دور_سلسہ_وار_مزاحمتی_ادوار}-الف میں ایک ایسا ہی برقی دور دکھایا گیا ہے۔اس برقی دور میں کُل تار کی مزاحمت \عددیء{R_{\textup{تار}}} ہے۔ اگر تار کی مزاحمت کو نظرانداز کیا جا سکے تو ہمیں شکل \حوالہ{شکل_مقناطیسی_دور_سلسہ_وار_مزاحمتی_ادوار}-ب ملتا ہے۔اس برقی دور میں برقی دباؤ \عددیء{v} کو مزاحمت \عددیء{R} تک بغیر کم کئے پہنچایا گیا ہے۔
\begin{figure}
\centering
\includegraphics{figMagneticCircuitsResistiveSeriesCircuit}
\caption{برقی دور میں تار کی مزاحمت کو نظر انداز کیا جاتا ہے۔}
\label{شکل_مقناطیسی_دور_سلسہ_وار_مزاحمتی_ادوار}
\end{figure}
%---------------------
\begin{figure}
\centering
\includegraphics{figMagneticCircuitsResistiveParallelCircuit}
\caption{برقی رو کم مزاحمت کے راستے  زیادہ ہوتی ہے}
\label{شکل_مقناطیسی_دور_متوازی_مزاحمتی_دور}
\end{figure}
%

شکل \حوالہ{شکل_مقناطیسی_دور_متوازی_مزاحمتی_دور}  میں ایک اور مثال دی گئی ہے۔ یہاں ہم دیکھتے ہیں کہ برقی رو اس راستے زیادہ ہوتی ہے جس کی مزاحمت کم ہو۔ لہٰذا اگر \عددیء{R_1 < R_2}ہو تو \عددیء{i_1>i_2} ہو گی۔

\حصہ{مقناطیسی دور حصہ اول}
مقناطیسی دور بالکل برقی دور کی طرح ہوتے ہیں۔ بس ان میں برقی دباؤ \عددیء{v} کی جگہ مقناطیسی دباؤ \عددیء{\tau} ، برقی رو \عددیء{i}  کی جگہ مقناطیسی بہاؤ \عددیء{\phi}  اور مزاحمت \عددیء{R} کی جگہ  ہچکچاہٹ  \عددیء{\Re} ہوتی ہے۔ لہٰذا ہم بالکل ایک برقی دور کی طرح ایک مقناطیسی دور بنا سکتے ہیں۔ ایسا ہی ایک دور شکل \حوالہ{شکل_مقناطیسی__مقناطیسی_سلسلہ_وار_دور}-الف میں دکھایا گیا ہے۔
\begin{figure}
\centering
\includegraphics{figMagneticCircuitsReluctanceSeriesCircuit}
\caption{مقناطیسی دور}
\label{شکل_مقناطیسی__مقناطیسی_سلسلہ_وار_دور}
\end{figure}
%
یہاں بھی کوشش یہی ہے کہ کسی طرح مقناطیسی دباؤ \عددیء{\tau} کو بغیر کم کئے ہچکچاہٹ \عددیء{\Re_a} تک پہنچایا جائے۔ عموما ً \عددیء{\Re_a} خلائی درز کی ہچکچاہٹ ہوتی ہے اور \عددیء{\Re_c} مقناطیسی مرکز کی۔ یہاں بھی اگر \عددیء{\Re_c} کو نظرانداز کرنا ممکن ہو تو ہمیں شکل \حوالہ{شکل_مقناطیسی__مقناطیسی_سلسلہ_وار_دور}-ب ملتا ہے جس میں مقناطیسی بہاؤ \عددیء{\phi} کو، بالکل اوہم کے قانون کی طرح، مساوات سے حل کیا جا سکتا ہے۔ یعنی
\begin{align}
\tau=\phi \Re_a
\end{align}
 اگر \عددیء{\Re_c} کو نظرانداز کرنا ممکن نہ ہو تب بالکل سلسلہ وار مزاحمتوں کی طرح ہم اس شکل میں دیئے گئے دو سلسلہ وار ہچکچاہٹوں کا مجموعہ ہچکچاہٹ  \عددیء{\Re_s} کو استعمال کر کے برقی رو کا حساب لگائیں گے، یعنی
\begin{align}
\Re_s&=\Re_a+\Re_c\\
\tau&=\phi \Re_s \label{مساوات_مقناطیسی_دور_مقناطیسی_اوہم_قانون}
\end{align}
	بالکل برقی مثال کی طرح، مقناطیسی دباؤ کو کم ہچکچاہٹ والے راستے سے اس جگہ پہنچایا جاتا ہے جہاں اس کی ضرورت ہو۔ مساوات \حوالہ{مساوات_مقناطیسی_دور_ہچکچاہٹ_کی_تعریف}  سے ہم دیکھتے ہیں کہ ہچکچاہٹ،  مقناطیسی مستقل \عددیء{\mu} سے منسلک ہے ۔\عددیء{\mu} کو عموما ً \عددیء{\mu=\mu_r \mu_0} لکھا جاتا ہے جہاں  \عددیء{\mu_0=4 \pi \times 10^{-7}} ہینری فی میٹر\حاشیہب{Henry per meter}  کے برابر ہے۔ لوہا،  کچھ دھاتیں اور چند جدید مصنوعی اشیاء  ایسی ہیں جن کی \عددیء{2000 < \mu_r < 80000} ہے۔ لہٰذا انہیں کو مقناطیسی دباؤ  ایک جگہ سے دوسری جگہ منتقلی کے لئے استعمال کیا جاتا ہے۔ البتہ \عددیء{\mu} کی مقدار اتنی نہیں ہے کہ اس سے بنی سلاخ کی ہچکچاہٹ ہر جگہ نظرانداز کی جا سکے۔ مساوات \حوالہ{مساوات_مقناطیسی_دور_ہچکچاہٹ_کی_تعریف}  سے ہم دیکھتے ہیں کہ ہچکچاہٹ کم سے کم کرنے کی خاطر رقبہ عمودی تراش زیادہ سے زیادہ رکھنی پڑتی ہے۔ لہٰذا عموما ً مقناطیسی دباؤ منتقل کرنے کے لئے ایک تار نہیں بلکہ خاصی زیادہ سطح عمودی تراش رکھنے والا راستہ  درکار ہوتا ہے جسے مقناطیسی مرکز\فرہنگ{مقناطیسی مرکز}\حاشیہب{magnetic core}\فرہنگ{magnetic core} کہتے ہیں۔برقی آلوں میں استعمال  مقناطیسی مرکز لوہے کی باریک چادر یا پتری\فرہنگ{پتری}\حاشیہب{laminations}\فرہنگ{laminations}  تہہ  در تہہ رکھ کر بنائی جاتی ہے۔ مقناطیسی مرکز کے بارے میں ہم حصہ \حوالہ{حصہ_مقناطیسی_دور_مقناطیسی_مادہ_کے_خصوصیات}  میں مزید معلومات حاصل کریں گے۔

\حصہ{کثافتِ مقناطیسی بہاؤ  اور مقناطیسی میدان کی شدت}\شناخت{حصہ_برقی_دور_کثافت_مقناطیسی_بہاو_اور_میدان}
حصہ \حوالہ{حصہ_برقی_دور_کثافت_برقی_رو_اور_میدان}  میں ہم نے برقی مثال دی۔ یہاں ہم مقناطیسی مثال پیش  کرتے ہیں۔ شکل \حوالہ{شکل_مقناطیسی__کثافت_مقناطیسی_بہاو_اور_شدت} میں ایک مقناطیسی مثال دکھائی گئی ہے۔یہاں مقناطیسی مرکز کی \عددیء{\mu_r = \infty} تصور کی گئی ہے لہٰذا اس مرکز کی ہچکچاہٹ \عددیء{\Re_c} صفر ہو گی۔ لہٰذا جیسے حصہ \حوالہ{حصہ_برقی_دور_کثافت_برقی_رو_اور_میدان}   میں تانبہ کی تار استعمال کی گئی تھی یہاں اسی طرح مقناطیسی مرکز کو مقناطیسی دباؤ \عددیء{\tau} ایک جگہ سے دوسری جگہ منتقل کرنے کے لئے استعمال کیا گیا ہے۔ اس شکل میں مقناطیسی دباؤ کو خلائی درز کی ہچکچاہٹ \عددیء{\Re_a} تک پہنچایا گیا ہے۔
\begin{figure}
\centering
\includegraphics{figMagneticCircuitsMagneticFluxDensityAndIntensity}
\caption{کثافتِ مقناطیسی بہاؤ اور مقناطیسی میدان کی شدت۔}
\label{شکل_مقناطیسی__کثافت_مقناطیسی_بہاو_اور_شدت}
\end{figure}

لہٰذا یہاں کُل ہچکچاہٹ صرف خلائی درز کی ہچکچاہٹ ہی ہے یعنی
\begin{align}
\Re_a=\frac{l_a}{\mu_0 A_z}
\end{align}
خلائی درز کے رقبہ عمودی تراش \عددیء{A_a} کو مرکز کے رقبہ عمودی تراش \عددیء{\Re_c} کے برابر لیا گیا ہے۔ یعنی 
\begin{align}
A_a=A_c=w b
\end{align}
اگر خلائی درز کی لمبائی \عددیء{l_a} مرکز کے رقبہ کے اطراف \عددیء{b} اور \عددیء{w} سے نہایت کم ہو یعنی \عددیء{l_a \ll b} اور \عددیء{l_a \ll w} تب ایسا کرنا ممکن ہوتا ہے۔ اس کتاب میں یہی تصور کیا جائے گا۔

مقناطیسی دباؤ کو یوں بیان کیا جاتا ہے
\begin{align}
\tau=N i
\end{align}
یعنی برقی تار کے چکر ضربِ ان میں برقی رو۔ لہٰذا مقناطیسی دباؤ کی اکائی ایمپیئر-چکر\فرہنگ{ایمپیئر-چکر}\حاشیہب{ampere-turn}\فرہنگ{ampere-turn}  ہے۔ بالکل حصہ \حوالہ{حصہ_برقی_دور_کثافت_برقی_رو_اور_میدان}   کی طرح ہم مساوات \حوالہ{مساوات_مقناطیسی_دور_مقناطیسی_اوہم_قانون} کو یوں لکھ سکتے ہیں۔
\begin{align}\label{مساوات_مقناطیسی_ڈور_بہاو_مساوی_دباو_بٹا_ہچکچاہٹ}
\phi_a=\frac{\tau}{\Re_a}
\end{align}
مقناطیسی بہاؤ کی اکائی ویبر\فرہنگ{ویبر}\حاشیہب{Weber}\فرہنگ{Weber}\حاشیہد{یہ اکائی جرمنی کے ولیم اڈورڈ ویبر کے نام ہے جن کا برقی و مقناطیسی میدان میں اہم کردار رہا ہے}  ہے اور ہچکچاہٹ کی اکائی ایمپیئر-چکر فی ویبر\حاشیہب{ampere-turn per weber} ہے۔  خلائی درز میں مقناطیسی بہاؤ \عددیء{\phi_a} اور مرکز میں مقناطیسی بہاؤ \عددیء{\phi_c} برابر ہیں۔ اس مساوات کو مساوات \حوالہ{مساوات_مقناطیسی_دور_ہچکچاہٹ_کی_تعریف}   کی مدد سے یوں لکھ سکتے ہیں۔
\begin{align}
\phi_a &=\tau \left(\frac{\mu_0 A_a}{l_a} \right) \nonumber \\
\intertext{یا}
\frac{\phi_a}{A_a}&=\mu_0 \left( \frac{\tau}{l_a} \right) \label{مساوات_مقناطیسی_دور_کثافت_بہاو_اوہم_قانون_سے}
\end{align}
	اس مساوات میں بائیں جانب مقناطیسی بہاؤ فی اکائی رقبہ کو کثافتِ مقناطیسی بہاؤ\فرہنگ{مقناطیسی بہاو!کثافت}\حاشیہب{magnetic flux density}\فرہنگ{magnetic flux!density} \عددیء{B_a} اور دائیں جانب برقی دباؤ فی اکائی لمبائی کو مقناطیسی میدان کی شدت\فرہنگ{مقناطیسی میدان!شدت}\حاشیہب{magnetic field intensity}\فرہنگ{magnetic field!intensity}  \عددیء{H_a} لکھا جا سکتا ہے۔یعنی
\begin{align}
B_a&=\frac{\phi_a}{A_a}\\
H_a&=\frac{\tau}{l_a}
\end{align}
کثافتِ مقناطیسی بہاؤ کی اکائی ویبر فی مربہ میٹر ہے جس کو ٹیسلہ\فرہنگ{ٹیسلہ}\فرہنگ{Tesla}\حاشیہد{Tesla:  یہ اکائی سربیا کے نِکولا ٹیسلہ کے نام ہے جنہوں نے بدلتی رو برقی طاقت عام کرنے میں اہم کردار ادا کیا}  کا نام دیا گیا ہے۔مقناطیسی میدان کی شدت کی اکائی ایمپیئر فی میٹر\حاشیہب{ampere per meter}  ہے۔ لہٰذا مساوات \حوالہ{مساوات_مقناطیسی_دور_کثافت_بہاو_اوہم_قانون_سے} کو ہم یوں لکھ سکتے ہیں۔
\begin{align}
B_a=\mu_0 H_a
\end{align}
جہاں متن سے واضح ہو کہ مقناطیسی میدان کی بات ہو رہی ہے وہاں مقناطیسی میدان کی شدت کو میدانی شدت\حاشیہب{field intensity} کہا جاتا ہے۔  شکل میں ہم دیکھتے ہیں کہ خلائی درز میں مقناطیسی بہاؤ کی سمت،  اکائی سمتیہ \عددیء{\az} کی الٹ سمت میں ہے لہٰذا ہم کثافتِ مقناطیسی بہاؤ کو \عددیء{\kvec{B_a}=-B_a \az} لکھ سکتے ہیں۔ اسی طرح خلائی درز میں مقناطیسی دباؤ  اکائی سمتیہ \عددیء{\az} کی الٹ سمت میں دباؤ ڈال رہی ہے لہٰذا ہم مقناطیسی دباؤ کی شدت کو \عددیء{\kvec{H_a}=-H_a \az} لکھ سکتے ہیں۔ لہٰذا اس مساوات کو یوں لکھا جا سکتا ہے۔
\begin{align}
\kvec{B_a}=\mu_0 \kvec{H_a}
\end{align}
اگر خلاء کی جگہ کوئی ایسے مادہ ہو جس کی ہو، تب ہم اس مساوات کو یوں لکھتے
\begin{align}
\kvec{B}=\mu \kvec{H}
\end{align}
%
\ابتدا{مثال}
شکل \حوالہ{شکل_مقناطیسی__کثافت_مقناطیسی_بہاو_اور_شدت} میں خلائی درز میں کثافتِ مقناطیسی بہاؤ \عددیء{0.1} ٹیسلہ درکار ہے۔مرکز کی \عددیء{\mu_r=\infty}  ہے اور خلائی درز کی لمبائی \عددیء{1} ملی میٹر ہے۔اگر  مرکز کے گرد برقی تار کے \عددیء{100} چکر ہوں تو ان میں درکار برقی رو معلوم کریں۔

حل:
\begin{align*}
\tau&=\phi \Re\\
N i & \phi \left(\frac{l}{\mu_0 A} \right)\\
\frac{\phi}{A}&=\frac{ N i \mu_0}{l}
\end{align*}
لہٰذا
\begin{align*}
0.1&=\frac{100 \times i \times 4 \pi  10^{-7}}{0.001}\\
i&=\frac{0.1 \times 0.001}{100 \times 4 \pi  10^{-7}}=\SI{0.79567}{\ampere}
\end{align*}
یعنی \عددیء{0.79567} ایمپیئر برقی رو سے خلائی درز میں \عددیء{0.1} ٹیسلہ کثافتِ مقناطیسی بہاؤ حاصل ہو جائے گی۔
\انتہا{مثال}
%
\حصہ{مقناطیسی دور حصہ دوم}
شکل \حوالہ{شکل_مقناطیسی__سادہ_مقناطیسی_دور_بغیر_درز} میں ایک سادہ مقناطیسی نظام دکھایا گیا ہے جس میں مرکز کی مقناطیسی مستقل کو محدود تصور کیا گیا ہے۔شکل میں مقناطیسی دباؤ  \عددیء{\tau=N i} مقناطیسی مرکز میں مقناطیسی بہاؤ \عددیء{\phi_c} کو جنم دیتی ہے۔ یہاں مرکز کا رقبہ عمودی تراش \عددیء{A_c}  ہر جگہ یکساں ہے اور مرکز  کی اوسط لمبائی \عددیء{l_c} ہے۔ مرکز میں مقناطیسی بہاؤ  کی سمت فلیمنگ\حاشیہب{Fleming's right hand rule} کے دائیں ہاتھ کے قانون  سے معلوم کی جا سکتی ہے۔  اس قانون کو دو طریقوں سے بیان کیا جا سکتا ہے۔
\begin{itemize}
\item
اگر ایک لچھے کو دائیں ہاتھ سے یوں پکڑا  جائے کہ ہاتھ کی چار انگلیاں لچھے میں برقی رو کی سمت میں لپٹی  ہوں تو انگوٹھا اُس مقناطیسی بہاؤ کی سمت میں ہوگا جو اس برقی رو کی وجہ سے وجود میں آئیگا۔
\item
اگر ایک تار جس میں برقی رو کا گزر ہو، کو دائیں ہاتھ سے یوں پکڑا جائے کہ انگوٹھا  برقی رو  کی سمت میں ہو تو باقی چار انگلیاں اُس مقناطیسی  رو ، جو اس برقی رو کی وجہ سے وجود میں آئے،  کی سمت میں لپٹی ہوں گی۔
\end{itemize}

ان دو بیانات میں پہلا بیان،  لچھے میں مقناطیسی بہاؤ کی سمت معلوم کرنے کے لئے زیادہ آسان ثابت ہوتا ہے جبکہ کسی ایک سیدھی تار کے گرد مقناطیسی بہاؤ کی سمت دوسرے بیان سے زیادہ آسانی سے معلوم کی جا سکتی ہے۔
\begin{figure}
\centering
\includegraphics{figMagneticCircuitsSimpleMagneticCircuitNoGap}
\caption{سادہ مقناطیسی دور}
\label{شکل_مقناطیسی__سادہ_مقناطیسی_دور_بغیر_درز}
\end{figure}
لہٰذا مرکز میں مقناطیسی بہاؤ  گھڑی کے سمت میں ہے۔ مقناطیسی بہاو کو  شکل میں تیر والے ہلکی سیاہی کے لکیر  سے ظاہر کیا گیا ہے۔ یہاں مرکز کی ہچکچاہٹ 
\begin{align}
\Re_c&=\frac{l_c}{\mu_c A_c}\\
\phi_c&=\frac{\tau}{\Re_c}=N i \left(\frac{\mu_c A_c}{l_c} \right)
\end{align}
اس طرح ہم سب متغیرات حاصل کر سکتے ہیں۔
%
\ابتدا{مثال}
شکل \حوالہ{شکل_مقناطیسی__درز_اور_ہچکچاہٹ}  میں ایک مقناطیسی مرکز دکھایا گیا ہے جہاں
\begin{align}
\text{کور}= \left\{ 
  \begin{array}{l l}
  h=\SI{20}{\centi\meter} & m=\SI{10}{\centi \meter}\\
 n=\SI{8}{\centi\meter} & w=\SI{2}{\centi \meter}\\
 l_a=\SI{1}{\milli\meter} & \mu_r =40000 \\
 \end{array} \right.
\end{align}
ہیں۔مرکز اور خلائی درز کی ہچکچاہٹیں حاصل کریں۔
\begin{figure}
\centering
\includegraphics{figMagneticCircuitsCoreWithGapAndReluctance}
\caption{خلائی درز اور مرکز کے ہچکچاہٹ}
\label{شکل_مقناطیسی__درز_اور_ہچکچاہٹ}
\end{figure}
حل:
\begin{align*}
b&=\frac{m-n}{2}=\frac{0.1-0.08}{2}=\SI{0.01}{\meter}\\
A_a&=A_c=bw=0.01 \times 0.02=\SI{0.0002}{\square \meter}\\
l_c&=2(h+n)-l_a=2(0.2+0.08)-0.001=\SI{0.559}{\meter}
\end{align*}
%
\begin{align*}
\Re_c&=\frac{l_c}{\mu_r \mu_0 A_c}=\frac{0.559}{40000 \times 4 \pi 10^{-7} \times 0.0002}=\SI{55598}{\ampere \cdot t \per \weber}\\
\Re_a&=\frac{l_a}{\mu_0 A_a}=\frac{0.001}{4 \pi 10^{-7} \times 0.0002}=\SI{3978358}{\ampere \cdot t \per \weber}
\end{align*}
ہم دیکھتے ہیں اگرچہ مرکز کی لمبائی خلائی درز کی لمبائی سے \عددیء{559} گنا زیادہ ہے تب بھی خلائی درز کی ہچکچاہٹ \عددیء{71} گنا زیادہ ہے یعنی \عددیء{\Re_a  \gg \Re_c} 
\انتہا{مثال}
%
\ابتدا{مثال}
شکل  \حوالہ{شکل_مقناطیسی_دور_سادہ_گھومتا_مشین} سے رجوع کریں۔اگر ایک خلائی درز \عددیء{5} ملی میٹر لمبا ہو اور گھومتے حصہ پر \عددیء{1000} چکر ہوں تو خلائی درز میں \عددیء{0.95} ٹیسلہ کثافتِ برقی بہاؤ حاصل کرنے کی خاطر درکار برقی رو معلوم کریں۔
\begin{figure}
\centering
\includegraphics{figMagneticCircuitsSimpleRotatingMachineOutline}
\caption{سادہ گھومنے والا مشین}
\label{شکل_مقناطیسی_دور_سادہ_گھومتا_مشین}
\end{figure}
حل:
	 اس شکل میں ایک گھومتے مشین، مثلاً موٹر، کی ایک سادہ شکل دکھائی گئی ہے۔ ایسے آلوں میں باہر کا حصہ ساکن رہتا ہے جس کو مشین کا ساکن حصہ کہتے ہیں اور اس ساکن حصہ کے اندر اس کا ایک حصہ گھومتا ہے جسے گھومتا حصہ کہتے ہیں۔ اس مثال میں ان دونوں حصوں کا  \عددیء{\mu_r=\infty}  ہے لہٰذا ان کی ہچکچاہٹ صفر ہے۔ مقناطیسی بہاؤ  ہلکی سیاہی کے لکیر سے ظاہر کی گئی ہے۔ یہ خلائی درز میں سے، ایک مکمل چکر کے دوران، دو مرتبہ گزرتی ہے۔ یہ دو خلائی درز ہر لحاظ سے ایک جیسے ہیں لہٰذا ان دونوں خلائی درز کی ہچکچاہٹ بھی برابر ہوں گی۔مزید یہ کہ ان خلائی درز کی ہچکچاہٹ سلسلہ وار ہیں۔شکل میں مقناطیسی بہاؤ کو گھومتے حصہ سے ساکن حصہ کی طرف، خلائی درز سے گزرتے دکھایا گیا ہے۔خلائی درز کی لمبائی \عددیء{l_a} بہت کم ہے لہٰذا خلائی درز کا عمودی رقبہ تراش \عددیء{A_a} وہی ہو گا جو گھومتے حصہ کا ہے یعنی \عددیء{A_c} 

  ایک خلائی درز کی ہچکچاہٹ
\begin{align*}
\Re_a=\frac{l_a}{\mu_0 A_a}=\frac{l_a}{\mu_0 \A_c}
\end{align*}
ہے۔لہٰذا کُل ہچکچاہٹ ہو گی
\begin{align*}
\Re_s=\Re_a=\Re_a=\frac{2 l_a}{\mu_0 A_c}
\end{align*}
یوں خلائی درز میں مقناطیسی بہاؤ \عددیء{\phi_a} اور کثافتِ مقناطیسی بہاؤ \عددیء{B_a} یہ ہوں گے۔
\begin{align*}
\phi_a&=\frac{\tau}{\Re_s}=\left(N i \right) \left (\frac{\mu_0 A_c}{2 l_a} \right)\\
B_a&=\frac{\phi_a}{A_a}=\frac{\mu_0 N i}{2 l_a}
\end{align*}
اس مساوات میں اعداد استعمال کرتے ہیں
\begin{align*}
0.95&=\frac{4 \pi 10^{-7} \times 1000 \times i}{2 \times 0.005}\\
i&=\frac{0.95 \times 2 \times 0.005}{ 4 \pi 10^{-7} \times 1000}=\SI{7.56}{\ampere}
\end{align*}
موٹر اور جنریٹروں کی خلاء میں تقریباً ایک ٹیسلہ کثافتِ برقی بہاؤ ہوتی ہے۔
\انتہا{مثال}

\حصہ{خود امالہ  ، مشترکہ امالہ  اور توانائی}
مقناطیسی بہاؤ کی، وقت کے ساتھ تبدیلی، برقی دباؤ  کو جنم دیتی ہے۔ لہٰذا  اگر شکل \حوالہ{شکل_مقناطیسی__کثافت_مقناطیسی_بہاو_اور_شدت}  کے مرکز میں مقناطیسی بہاؤ تبدیل ہو رہی ہو تو اس کی وجہ سے اس کے لچھے میں برقی دباؤ پیدا ہوگی جوکہ اس لچھے کے سروں پر نمودار ہوگی۔ اِس طرح پیدا ہونے والی برقی دباؤ کو امالی برقی دباؤ\فرہنگ{امالی برقی دباو}\حاشیہب{induced voltage}\فرہنگ{induced voltage}  کہتے ہیں۔ قانونِ فیراڈے\فرہنگ{فیراڈے!قانون}\حاشیہب{Faraday's law}\فرہنگ{Faraday's law}  کے تحت\حاشیہد{مائکل فیراڈے انکلستانی سائنسدان تھے جنہوں نے محرک برقی دباؤ دریافت کی}
\begin{align}\label{مساوات_مقناطیسی_دور_فیراڈے_قانون}
e=N \frac{\partial \phi}{\partial t} =\frac{\partial \lambda}{\partial t}
\end{align}
اس مساوات میں ہم لچھے میں، وقت کے ساتھ تبدیل ہونے والی، مقناطیسی بہاؤ کو \عددیء{\phi} سے ظاہر کر رہے ہیں۔\عددیء{N \phi} کو لچھے کی اِرتَباطِ بہاؤ\فرہنگ{ارتباط بہاو}\حاشیہب{flux linkage} \عددیء{\lambda}  کہتے ہیں جس کی اکائی ویبر-چکر\فرہنگ{ویبر-چکر}\حاشیہب{weber-turn}  ہے۔ اس امالی برقی دباؤ  کی سمت کا تعین یوں کیا جاتا ہے کہ اگر دیئے گئے لچھے کی سروں کو کسرِ دور\فرہنگ{کسر دور}\حاشیہب{short circuit}   کیا جائے تو اِس میں برقی رو اُس سمت میں ہو گی جس میں مقناطیسی بہاؤ کی تبدیلی کو روکا جا سکے۔ 

جن مقناطیسی دوروں میں مقناطیسی مستقل \عددیء{\mu}  کو اٹل مقدار تصور کیا جا سکے یا جن میں خلائی درز کی ہچکچاہٹ مرکز کی ہچکچاہٹ سے بہت زیادہ ہو یعنی \عددیء{\Re_a \gg \Re_c} ، ان حالات میں ہم لچھے کی  امالہ\فرہنگ{امالہ}\حاشیہب{inductance}\فرہنگ{inductance} \عددیء{L}  کو یوں بیان کرتے ہیں۔
\begin{align}\label{مساوات_مقناطیسی_دور_خود_امالہ_تعریف}
L=\frac{\lambda}{i}
\end{align}

امالہ کی اکائی ویبر-چکر فی ایمپیئر ہے جس کو ہینری\فرہنگ{Henry}\حاشیہب{Henry}\فرہنگ{Henry} \عددیء{H} کا نام\حاشیہد{امریکی سائنسدان جوزف ہینری جنہوں نے مائکل فیراڈے سے علیحدہ  طور پر محرک برقی دباؤ دریافت کی} دیا گیا ہے۔ لہٰذا
\begin{align}
L=\frac{N \phi}{i}=\frac{N B_c A_c}{i}=\frac{N^2 \mu_0 A_a}{l_a}
\end{align}
%
\ابتدا{مثال}
شکل \حوالہ{شکل_مقناطیسی__کثافت_مقناطیسی_بہاو_اور_شدت} میں اگر \عددیء{b=\SI{5}{\centi \meter},w=\SI{4}{\centi\meter},l_a=\SI{3}{\milli \meter}} جبکہ لچھے کے \عددیء{1000} چکر اور مرکز کے گرد اوسط لمبائی \عددیء{l_c=\SI{30}{\centi\meter}} ہو تب ان دو صورتوں میں لچھے کی امالہ معلوم کریں۔
\begin{itemize}
\item
مرکز کی \عددیء{\mu_r = \infty} ہے۔
\item
مرکز کی \عددیء{\mu_r = 500} ہے۔
\end{itemize}

حل:
	پہلی صورت میں مرکز کی \عددیء{\mu_r=\infty} ہونے کی وجہ سے مرکز کی ہچکچاہٹ نظرانداز کی جا سکتی ہے۔یوں
\begin{align*}
L&=\frac{N^2 \mu_0 w b}{l_a}\\
&=\frac{1000^2 \times 4 \pi 10^{-7} \times 0.04 \times 0.05}{0.003}\\
&=\SI{0.838}{\henry}
\end{align*}
	دوسری صورت میں \عددیء{\mu_r=500} ہے۔یوں مرکز کی ہچکچاہٹ صفر نہیں۔خلاء اور مرکز کی ہچکچاہٹ پہلے دریافت کرتے ہیں
\begin{align*}
\Re_a&=\frac{l_a}{\mu_0 w b}=\frac{0.003}{4\pi 10^{-7} \times 0.04 \times 0.05}=\SI{1193507}{\ampere \cdot t \per \weber}\\
\Re_c&=\frac{l_c}{\mu_r \mu_0 w b}=\frac{0.3}{500 \times 4\pi 10^{-7} \times 0.04 \times 0.05}=\SI{238701}{\ampere \cdot t \per \weber}
\end{align*}
لہٰذا
\begin{align*}
\phi&=\frac{N i}{\Re_a+\Re_c}\\
\lambda &= N \phi = \frac{N^2 i}{\Re_a+\Re_c}\\
L&=\frac{\lambda}{i}=\frac{N^2}{\Re_a+\Re_c}=\frac{1000^2}{1193507+238701}=\SI{0.698}{\henry}
\end{align*}
\انتہا{مثال}
%
\ابتدا{مثال}
شکل \حوالہ{شکل_مقناطیسی_ادوار_پیچدار_لچھا} میں ایک پیچدار لچھا\فرہنگ{لچھا!پیچدار}\حاشیہب{spiral coil} دکھایا گیا ہے جس کی تفصیل یوں ہے

\عددیء{N=11, r=\SI{0.49}{\meter},l=\SI{0.94}{\meter}}

ایسے پیچدار لچھے کی بیشتر مقناطیسی بہاؤ لچھے کے اندر محوری سمت میں ہوتی ہے۔لچھے کے باہر مقناطیسی بہاؤ کی مقدار قابلِ نظرانداز ہوتی ہے۔یوں لچھے کے اندر محوری جانب مقناطیسی شدت
\begin{align*}
H=\frac{N i}{l}
\end{align*}
ہوتی ہے۔اس لچھے کی خود امالہ حاصل کریں۔
\begin{figure}
\centering
\includegraphics{figMagneticCircuitsCoil}
\caption{پیچدار لچھا}
\label{شکل_مقناطیسی_ادوار_پیچدار_لچھا}
\end{figure}
حل:
\begin{align*}
B&=\mu_0 H=\frac{\mu_0 N i}{l}\\
\phi&=B  \pi r^2=\frac{\mu_0 N i \pi r^2}{l}\\ 
\lambda&=N \phi =\frac{\mu_0 N^2 i \pi r^2}{l}\\ 
L&=\frac{\lambda}{i}=\frac{\mu_0 N^2 \pi r^2}{l}
\end{align*} 
یوں
\begin{align*}
L=\frac{4 \pi 10^{-7} \times 11^2 \times \pi  \times 0.49^2}{0.94}=\SI{122}{\micro \henry}
\end{align*}
یہ پیچدار لچھا میں نے  \عددیء{3000} کلو گرام لوہا گلانے والی بھٹی میں استعمال کیا ہے۔
\انتہا{مثال}
%
شکل \حوالہ{شکل_مقناطیسی_ادوار_دو_لچھے_ایک_درز} میں دو لچھے والا ایک مقناطیسی دور دکھایا گیا ہے۔ ایک لچھے کے  \عددیء{N_1} چکر ہیں اور اس میں برقی رو \عددیء{i_1} ہے اور دوسرا لچھا \عددیء{N_2} چکر کا ہے اور اس میں برقی  رو \عددیء{i_2} ہے۔ دونوں لچھوں میں برقی رو کی سمتیں یوں ہیں کہ اِن  دونوں کا مقناطیسی دباؤ جمع ہو۔ یوں اگر مرکز کے امالہ کو نظرانداز کیا جائے تو ہم مقناطیسی بہاؤ \عددیء{\phi} کے لئے لکھ سکتے ہیں
\begin{align}
\phi=\left (N_1 i_1 +N_2 i_2 \right ) \frac{\mu_0 A_a}{l_a}
\end{align} 
%
\begin{figure}
\centering
\includegraphics{figMagneticCircuitsTwoCoilsWithGap}
\caption{دو لچھے والا مقناطیسی دور۔}
\label{شکل_مقناطیسی_ادوار_دو_لچھے_ایک_درز}
\end{figure}
یہاں \عددیء{\phi} دونوں لچھوں کے مجموعی مقناطیسی دباؤ یعنی \عددیء{N_1 i_1+N_2 i_2} سے پیدا ہونے والا مقناطیسی بہاؤ ہے۔ اس مقناطیسی بہاؤ  کی ان  لچھوں  کے ساتھ  اِرتَباط کو یوں لکھا جا سکتا ہے۔
\begin{align}
\lambda_1=N_1 \phi=N_1^2  \frac{\mu_0 A_a}{l_a} i_1 +N_1 N_2  \frac{\mu_0 A_a}{l_a} i_2
\end{align}
اس کو یوں لکھا جا سکتا ہے
\begin{align}\label{مساوات_مقناطیسی_دور_ارتباط_دو_لچھے}
\lambda_1 = L_{11} i_1+L_{12} i_2
\end{align}
جہاں
\begin{align}
L_{11}&=N_1^2  \frac{\mu_0 A_a}{l_a}\\
L_{12}&=N_1 N_2  \frac{\mu_0 A_a}{l_a}
\end{align}
ہیں۔ یہاں \عددیء{L_{11}} پہلے لچھے کی  خود امالہ\فرہنگ{خود امالہ}\حاشیہب{self inductance}\فرہنگ{self inductance}  ہے اور  \عددیء{L_{11} i_1} اِس لچھے کی اپنے برقی رو \عددیء{i_1} سے پیدا مقناطیسی بہاؤ  کے ساتھ  اِرتَباطِ بہاؤ ہے جسے خود اِرتَباطِ بہاؤ\فرہنگ{خود ارتباط بہاو}\حاشیہب{self flux linkage}\فرہنگ{self flux linkage} کہتے ہیں۔\عددیء{L_{12}}  اِن دونوں لچھوں  کا  مشترکہ امالہ\فرہنگ{مشترکہ امالہ}\حاشیہب{mutual inductance}\فرہنگ{mutual inductance} ہے اور  \عددیء{L_{12} i_2}  لچھا نمبر-1  کے ساتھ برقی رو  \عددیء{i_2} کی وجہ سے پیدا کردہ مقناطیسی بہاؤ  کا اِرتَباطِ بہاؤ  ہے جسے مشترکہ اِرتَباطِ بہاؤ\فرہنگ{مشترکہ ارتباط امالہ}\حاشیہب{mutual flux linkage}\فرہنگ{mutual flux linkage}  کہتے ہیں ۔ بالکل اسی طرح ہم دوسرے لچھے کے لئے لکھ سکتے ہیں
\begin{align}
\lambda_2&=N_2 \phi=N_2 N_1 \frac{\mu_0 A_a}{l_a} i_1+N_2^2 \frac{\mu_0 A_a}{l_a} i_2  \nonumber \\
&=L_{21} i_1+L_{22} i_2 \label{مساوات_مقناطیسی_دور_دوسرے_لچھے_کی_ارتباط}
\end{align}
جہاں
\begin{align}
L_{22}&=N_2^2 \frac{\mu_0 A_a}{l_a}\\
L_{21}&=L_{12}=N_2 N_1 \frac{\mu_0 A_a}{l_a} \label{مساوات_مقناطیسی_دور_مشترکہ_امالہ_یکساں}
\end{align}
ہیں۔\عددیء{L_{22}} دو نمبر لچھے  کی خود امالہ اور  \عددیء{L_{21}=L_{12}} ان  دو لچھوں کی مشترکہ امالہ ہے۔ یہاں یہ واضح کرنا ضروری ہے کہ امالہ کا تصور اس وقت کارآمد ہوتا جب ہم مقناطیسی مستقل \عددیء{\mu}  کو اٹل تصور کر سکیں۔

مساوات \حوالہ{مساوات_مقناطیسی_دور_خود_امالہ_تعریف}  کو مساوات \حوالہ{مساوات_مقناطیسی_دور_فیراڈے_قانون}  میں استعمال کریں تو 
\begin{align}
e=\frac{\partial \lambda}{\partial t}=\frac{ \partial \left (N \phi \right)}{\partial t}=\frac{\partial \left( L i\right) }{\partial t}
\end{align}
اگر امالہ مقررہ ہو جیسا کہ ساکن آلوں میں ہوتا ہے تب ہمیں  امالہ کی جانی پہچانی مساوات ملتی ہے 
\begin{align}
e=L \frac{\partial i}{\partial t}
\end{align}
مگر اگر امالہ بھی تبدیل ہو جیسا کہ موٹروں اور جنریٹروں میں ہوتا ہے تب
\begin{align}
e= L \frac{\partial i}{\partial t} + i \frac{\partial L}{\partial t}
\end{align}
توانائی\فرہنگ{توانائی}\حاشیہب{energy}\فرہنگ{energy}  کی اکائی جاؤل\فرہنگ{جاؤل}\حاشیہب{Joule}\فرہنگ{Joule} \عددیء{J}\حاشیہد{جیمس پریسقوٹ جاؤل انگلستانی سائنسدان جنہوں نے حرارت اور میکانی کام کا رشتہ دریافت کیا} ہے اور طاقت\فرہنگ{طاقت}\حاشیہب{power}\فرہنگ{power}  کی اکائی\حاشیہد{سکاٹلینڈ کے جیمز واٹ جنہوں نے بخارات پر چلنے والے انجن پر کام کیا} جاؤل فی سیکنڈ یا واٹ\فرہنگ{واٹ}\حاشیہب{Watt}\فرہنگ{Watt} \عددیء{W}  ہے۔

اس کتاب میں توانائی یا کام کو \عددیء{W} سے ظاہر کیا جائے گا مگر طاقت کی اکائی واٹ \عددیء{W} کے لئے بھی ہی کی علامت استعمال ہوتی ہے۔امید کی جاتی ہے کہ اس سے غلطی پیش نہیں آئے گی اور استعمال کو دیکھ کر یہ فیصلہ کرنا کہ اس کا کون سا مطلب لیا جا رہا ہے دشوار نہ ہوگا۔

وقت کے ساتھ توانائی کی شرح کو طاقت کہتے ہیں لہٰذا کسی لچھے کے لئے ہم لکھ سکتے ہیں
\begin{align}
p=\frac{\dif W}{\dif t} = e i = i \frac{\partial \lambda}{\partial t}
\end{align} 
لہٰذا ایک مقناطیسی دور میں  \عددیء{t_1} سے \عددیء{t_2} تک کے وقفے میں مقناطیسی توانائی میں تبدیلی کو تکمل کے ذریعہ یوں حاصل کیا جا سکتا ہے۔
\begin{align}
\Delta W = \int_{t1}^{t2} p \dif t =\int_{\lambda1}^{\lambda2} i \dif \lambda
\end{align}
اگر مقناطیسی دور میں ایک ہی لچھا ہو اور اس دور میں امالہ اٹل ہو تب
\begin{align}
\Delta W = \int_{\lambda1}^{\lambda2} i \dif \lambda=\int_{\lambda1}^{\lambda2} \frac{\lambda}{L} \dif \lambda=\frac{1}{2 L} \left(\lambda_2^2-\lambda_1^2 \right)
\end{align}

	اگر ہم لمحہ \عددیء{t_1} پہ \عددیء{\lambda_1=0} تصور کریں تب ہم کسی دیئے گئے \عددیء{\lambda} پہ مقناطیسی توانائی کو یوں لکھ سکتے ہیں
\begin{align}
\Delta W=\frac{\lambda^2}{2L}=\frac{L i^2}{2}
\end{align}

\حصہ{مقناطیسی مادہ کے خصوصیات}\شناخت{حصہ_مقناطیسی_دور_مقناطیسی_مادہ_کے_خصوصیات}
مقناطیسی دوروں میں مرکز استعمال کرنے سے دو طرح کے فوائد حاصل ہوتے ہیں۔ مرکز کے استعمال سے ایک تو کم مقناطیسی دباؤ سے زیاد مقناطیسی بہاؤ پیدا کی جا سکتی ہے اور دوسری، مقناطیسی بہاؤ کو اپنی مرضی کے راستوں پابند کیا جاسکتا ہے۔ ٹرانسفارمروں میں مرکز کو استعمال کر کے مقناطیسی بہاؤ کو اِس طرح پابند کیا جاتا ہے کہ جو مقناطیسی بہاؤ ایک لچھے سے گزرتا ہے، وہی مقناطیسی بہاؤ، سارا کا سارا، باقی لچھوں سے بھی گزرتا ہے۔ موٹروں میں مرکز کو استعمال کرکے مقناطیسی بہاؤ کو یوں پابند کیا جاتا ہے کہ زیادہ سے زیادہ قوت پیدا ہو جبکہ جنریٹروں میں اسے زیادہ سے زیادہ برقی دباؤ حاصل کرنے کی نیت سے پابند کیا جاتا ہے۔
\begin{figure}
\centering
\includegraphics{figMagneticCircuitsBHSingleLoop}
\caption{$BH$   خطوط یا مقناطیسی چال کے دائرے}
\label{شکل_مقناطیسی_چال}
\end{figure}
مقناطیسی اشیاء کی \عددیء{B} اور \عددیء{H} کے تعلق کو گراف کے ذریعہ سے پیش کیا جاتا ہے۔ لوہا نما مقناطیسی اشیاء کی \عددیء{B-H}  گراف شکل \حوالہ{شکل_مقناطیسی_چال}-الف میں دکھائی گئی ہے۔ایک لوہا نما مقناطیسی شہ جس میں کسی قسم کی مقناطیسی اثر نہ ہو کو نقطہ \عددیء{a} سے ظاہر کیا گیا ہے۔اس نقطہ پر
\begin{gather}
\begin{aligned}
H_a&=0\\
B_a&=0
\end{aligned}
\end{gather}
ہیں۔

	ایسی شہ کو لچھے میں رکھ کر اس پر مقناطیسی دباؤ لاگو کی جا سکتی ہے۔ مقناطیسی میدان کی شدت \عددیء{H}  لاگو کرنے سے لوہا نما مقناطیسی شہ میں کثافتِ مقناطیسی بہاؤ  \عددیء{B} پیدا ہوگی۔میدانی شدت بڑھانے سے کثافتِ مقناطیسی بہاؤ بھی بڑھے گی۔اس عمل کو نقطہ  \عددیء{a} سے شروع ایک نوکدار خط سے دکھلایا گیا ہے۔میدانی شدت کو نقطہ \عددیء{b}  تک بڑھایا گیا ہے جہاں یہ مقداریں  \عددیء{H_b} اور \عددیء{B_b} ہیں۔

	اگر اس نقطہ تک پہنچنے کے بعد میدانی شدت کم کی جائے تو دیکھا یہ گیا ہے کہ واپسی کی خط مختلف راستہ اختیار کرتی ہے۔یوں نقطہ  \عددیء{b} سے اگر میدانی شدت کم کرتے کرتے صفر کی جائے تو لوہا نما شہ کی کثافتِ مقناطیسی بہاؤ کم ہو کر نقطہ \عددیء{c} پر آ پہنچتی ہے۔نقطہ \عددیء{b} سے نقطہ \عددیء{c} تک نوکدار خط اس عمل کو دکھلا رہی ہے۔اس نقطہ پر بیرونی میدانی شدت صفر ہے لیکن لوہا نما شہ کی کثافتِ مقناطیسی بہاؤ صفر نہیں۔یہ اب ایک مقناطیس بن گیا ہے جس کی کثافتِ مقناطیسی بہاؤ  \عددیء{B_c} ہے۔اس مقدار کو بقایا کثافتِ مقناطیسی بہاؤ\فرہنگ{کثافت مقناطیسی بہاؤ!بقایا}\حاشیہب{magnetic flux!residual}\فرہنگ{residual magnetic flux}  کہتے ہیں۔مصنوعی مقناطیس اسی طرح بنائے جاتے ہیں۔

اگر یہاں سے میدانی شدت منفی سمت میں بڑھائی جائے تو \عددیء{B} کم ہوتے ہوتے آخر کار ایک بار پھر صفر ہو جاتی ہے۔اس نقطہ کو \عددیء{d} سے ظاہر کیا گیا ہے۔مقناطیسیت ختم کرنے کے لئے درکار میدانی شدت کی مقدار  \عددیء{\abs{H_d}} کو مقناطیسیت ختم کرنے والی شدت یا خاتم شدت\فرہنگ{مقناطیس!خاتم شدت}\حاشیہب{coercivity}\فرہنگ{coercivity} کہتے ہیں۔

منفی سمت میں میدانی شدت بڑھاتے نقطہ \عددیء{e} حاصل ہوتا ہے جہاں سے منفی سمت کی میدانی شدت کی مقدار ایک بار پھر کم کی جاتی ہے۔یوں نقطہ \عددیء{f} حاصل ہوتا ہے جہاں میدانی شدت صفر ہونے کے باوجود کثافتِ مقناطیسی بہاؤ صفر نہیں۔اس نقطہ پر لوہا نما شہ اُلٹ سمت میں مقناطیس بن چکا ہے اور \عددیء{B_f} بقایا کثافتِ مقناطیسی بہاؤ ہے۔اسی طرح اس جانب مقناطیسیت ختم کرنے کی شدت \عددیء{\abs{H_g}} ہے۔

اگر لوہا نما شہ پر باری باری مثبت اور منفی  یکساں میدانی شدت کئی بار لاگو کی جائے تو اس کی \عددیء{B-H}  کی خط ایک بند دائرہ کی شکل اختیار کر لیتی ہے جسے مقناطیسی چال کا دائرہ\فرہنگ{مقناطیس!چال کا دائرہ}\حاشیہب{hysteresis loop}\فرہنگ{hysteresis loop}  کہتے ہیں۔یہی شکل کے حصہ الف میں دکھائی گئی ہے۔

شکل \حوالہ{شکل_مقناطیسی_چال}-الف میں نقطہ  \عددیء{a} سے نقطہ  \عددیء{b} پہنچنے کے بعد اگر میدانی شدت مزید بڑھائی جائے اور پھر مقناطیسی چال حاصل کی جائے تو شکل کے حصہ با کا بیرونی بند دائرہ ملتا ہے۔حصہ الف کی مقناطیسی چال یہاں اندرونی دائرہ سے دکھائی گئی ہے۔

شکل \حوالہ{شکل_مقناطیسی_چال}   کی طرح کے خطوط کی چونچوں (یعنی زیادہ سے زیادہ مقدار واضح کرنے والے نکتوں) میں سے اگر ایک خط گزاری جائے تو شکل \حوالہ{شکل_مقناطیسی_ادوار_ایم_پانچ_پتری_کا_خط}  حاصل ہوتی ہے۔یہ شکل ٹرانسفارمروں میں استعمال ہونے والی  \عددیء{0.3048}  ملی میٹر موٹی \عددیء{M5} پتری کا گراف ہے۔ اس خط میں موجود مواد جدول \حوالہ{جدول_مقناطیسی_ادوار_کثافت_بہاو_بالمقابل_شدت}  میں بھی دیا گیا ہے۔عموماً مسائل اس خط میں موجود مواد سے حل ہوتے ہیں۔دھیان رہے کہ اس خط میں \عددیء{H}  کا پیمانہ لاگ\حاشیہب{log} میں دکھایا گیا ہے۔

لوہا نما مقناطیسی اشیاء پر لاگو مقناطیسی شدت بڑھانے سے کثافتِ مقناطیسی بہاؤ بڑھنے کی شرع بتدریج کم ہوتی جاتی ہے حتیٰ کہ آخر کار یہ شرح خلاء کی شرح  \عددیء{\mu_0} رہ جاتی ہے یعنی
\begin{align}
\frac{\Delta B}{\Delta H}=\mu_0
\end{align}
اس اثر کو سیرابیت\فرہنگ{سیرابیت}\حاشیہب{saturation}\فرہنگ{saturation} کہتے ہیں۔یہ شکل \حوالہ{شکل_مقناطیسی_ادوار_ایم_پانچ_پتری_کا_خط}  میں واضح ہے۔
\begin{figure}
\centering
\includegraphics{figMagneticCircuitsM5curve}
\caption{$M5$ سٹیل کی $0.3048$ ملی میٹر موٹی پتری کا خط۔ میدانی شدت کا پیمانہ لاگ ہے۔}
\label{شکل_مقناطیسی_ادوار_ایم_پانچ_پتری_کا_خط}
\end{figure}
%
گراف کو دیکھا جائے تو \عددیء{B} کے کسی ایک متعین مقدار \عددیء{H} کے لئےکے دو ممکنہ مقدار ہیں۔ اگر مقناطیسی بہاؤ بڑھ رہا ہو تو، گراف میں نیچے سے اُوپر جانے والی لکیر، اِس میں \عددیء{B} اور \عددیء{H} کے تعلق کو پیش کرتی ہے اور اگر مقناطیسی بہاؤ کم ہو رہا ہو تو، اوپر سے نیچے آنے والی لکیر، اِس تعلق کو پیش کرتی ہے۔  چونکہ \عددیء{\mu=B/H} ، لہٰذا \عددیء{B} کے  مقدار تبدیل ہونے سے \عددیء{\mu} بھی تبدیل ہوتی ہے۔ باوجود اِس کے ہم مقناطیسی دوروں میں یہ تصور کرتے ہیں کہ \عددیء{\mu} ایک مقررہ ہے۔ یہ تصور کر لینے سے عموماً جواب پر زیادہ اثر نہیں پڑتا۔
%
\begin{table}
\begin{tabular}{l l l l   l l l l   l l l l}
$H$&$B$&$H$&$B$&$H$&$B$&$H$&$B$&$H$&$B$&$H$&$B$\\
\hline\\
9000&1.998&1000&1.852&           200&1.720 &30&1.480               &9&0.700&  0&0.000    \\
10000&2.000&2000&1.900&         300&1.752 &40&1.540           &10&0.835&  2&0.040    \\
20000&2.020&3000&1.936&         400&1.780 &50&1.580          &11.22&1.000&  3&0.095    \\
30000&2.040& 4000&1.952&        500&1.800 &60&1.601         &12.59&1.100 &  4&0.160    \\
40000&2.048&5000&1.968&         600&1.810 &70&1.626          &14.96&1.200&   5&0.240    \\
50000&2.060&6000&1.975&         700&1.824 &80&1.640         &17.78&1.300&  6&0.330    \\
60000&2.070&7000&1.980&         800&1.835  &90&1.655         &20&1.340&  7&0.440    \\
 70000&2.080&8000&1.985&        900&1.846 &100&1.662          &23.77&1.400& 8&0.560    \\
\hline
\end{tabular}
\caption{مقناطیسی بہاو بالمقابل شدت}
\label{جدول_مقناطیسی_ادوار_کثافت_بہاو_بالمقابل_شدت}
\end{table}
%
\ابتدا{مثال}
شکل \حوالہ{شکل_مقناطیسی_ادوار_ایم_پانچ_پتری_کا_خط}  یا اس کے مساوی جدول \حوالہ{جدول_مقناطیسی_ادوار_کثافت_بہاو_بالمقابل_شدت} میں دیئے گئے مواد کو استعمال کرتے ہوئے شکل \حوالہ{شکل_مقناطیسی__کثافت_مقناطیسی_بہاو_اور_شدت}  کی خلاء میں ایک ٹیسلہ اور دو ٹیسلہ کثافتِ  مقناطیسی بہاؤ حاصل کرنے کے لئے درکار برقی رو معلوم کریں۔اس شکل میں
\begin{align*}
b=\SI{5}{\centi\meter},w=\SI{4}{\centi\meter},l_a=\SI{3}{\milli\meter},l_c=\SI{30}{\centi\meter},N=1000
\end{align*}
ہیں۔مرکز اور خلاء کی رقبہ عمودی تراش برابر لیں۔

حل: ایک ٹیسلہ کے لئے۔

 فہرست  سے ہم دیکھتے ہیں کہ مرکز میں \عددیء{1} ٹیسلہ  حاصل کرنے کے لئے  مرکز کو \عددیء{11.22}  ایمپیئر-چکر فی \عددیء{H} میٹر  درکار ہے۔یوں \عددیء{30} سم لمبے مرکز کو \عددیء{0.3\times 11.22=3.366}  ایمپیئر چکر درکار ہیں۔

خلاء کو
\begin{align*}
H=\frac{B}{\mu_0}=\frac{1}{4\pi 10^{-7}}=\num{795671}
\end{align*}
ایمپیئر-چکر فی میٹر درکار ہیں۔لہٰذا \عددیء{ 3 } ملی میٹر لمبی خلاء کو \عددیء{0.003 \times 795671=2387} ایمپیئر چکر درکار ہیں۔یوں کُل ایمپیئر-چکر \عددیء{3.366+2387=2390.366} ہیں جن سے 
\begin{align*}
i=\frac{2390.366}{1000}=\SI{2.39}{\ampere}
\end{align*}	
حاصل ہوتی ہے۔

حل: دو ٹیسلہ کے لئے۔

فہرست سے ہم دیکھتے ہیں کہ مرکز میں \عددیء{2} ٹیسلہ  حاصل کرنے کے لئے  مرکز کو \عددیء{10000} ایمپیئر-چکر فی میٹر \عددیء{H} درکار ہے۔یوں \عددیء{30} سم لمبے مرکز کو \عددیء{0.3 \times 10000=3000} ایمپیئر چکر درکار ہیں۔خلاء کو
\begin{align*}
H=\frac{B}{\mu_0}=\frac{2}{4\pi 10^{-7}}=\num{1591342}
\end{align*}
ایمپیئر-چکر فی میٹر درکار ہیں۔لہٰذا \عددیء{3} ملی میٹر لمبی خلاء کو  \عددیء{0.003 \times 1591342=4774}  ایمپیئر چکر درکار ہیں۔یوں کُل ایمپیئر-چکر \عددیء{3000+4774=7774}	ہیں جن سے 
\begin{align*}
i=\frac{7774}{1000}=\SI{7.774}{\ampere}
\end{align*}
حاصل ہوتی ہے۔

اس مثال میں مقناطیسی سیرابیت کے اثرات واضح ہیں۔ 
\انتہا{مثال}
%
\حصہ{ہیجان شدہ لچھا}
بدلتی رو میں برقی دباؤ اور مقناطیسی بہاؤ سائن نما ہوتے ہیں یعنی یہ وقت کے ساتھ \عددیء{\sin \omega t} یا \عددیء{\cos \omega t} کا تعلق رکھتے ہیں۔ اِس سبق میں ہم بدلتی رو سے لچھے کو ہیجان کرنا اور اس سے نمودار ہونے والے برقی توانائی  کے ضیاع  کا تذکرہ  کریں گے۔ ہم فرض کرتے ہیں کہ مرکز میں کثافتِ مقناطیسی بہاؤ 
\begin{align}
B=B_0 \sin \omega t
\end{align}
یوں مرکز میں بدلتا مقناطیسی بہاؤ \عددیء{\varphi}
\begin{align}
\varphi=A_c B=A_c B_0 \sin \omega t=\phi_0 \sin \omega t
\end{align}
ہے۔اس مساوات میں مقناطیسی بہاؤ کا حیطہ  \عددیء{\mp\phi_0} اور \عددیء{B} کا حیطہ \عددیء{\mp B_0} کے مابین تبدیل ہوتے ہیں۔\عددیء{A_c} مرکز کا رقبہ عمودی تراش ہے جو ہر جگہ یکساں ہے ۔\عددیء{\omega = 2 \pi f} ہے جہاں \عددیء{f} تعدد ہے۔

فیراڈے کے قانون یعنی مساوات \حوالہ{مساوات_مقناطیسی_دور_فیراڈے_قانون}  کے تحت اس مقناطیسی بہاؤ کی وجہ سے لچھے میں \عددیء{e(t)} برقی دباؤ  پیدا ہو گی۔
\begin{gather}
\begin{aligned}
e(t)&=\frac{\partial \lambda}{\partial t}\\
&=\omega N \phi_0 \cos \omega t \\
&=\omega N A_c B_0 \cos \omega t\\
&=E_0 \cos \omega t
\end{aligned}
\end{gather}
جس کا حیطہ
\begin{align}
E_0=\omega N \phi_0=2 \pi f N A_c B_0
\end{align}
ہے۔\عددیء{e(t)} کو امالی برقی دباؤ\فرہنگ{امالی برقی دباؤ}\حاشیہب{induced voltage}\فرہنگ{induced voltage} کہتے ہیں۔ 

ہم بدلتی رو مقداروں کے مربع کی اوسط کے جزر  میں دلچسپی رکھتے ہیں۔یہی ان مقداروں کی موثر\فرہنگ{موثر}\حاشیہب{root mean square, rms}\فرہنگ{rms} قیمت ہوتی ہے۔ جیسا صفحہ \حوالہصفحہ{مساوات_بنیادی_سائن_نما_کی_موثر_قیمت} پر مساوات \حوالہ{مساوات_بنیادی_سائن_نما_کی_موثر_قیمت}  میں دیکھا گیا ہے، ایک سائن نما  موج کی موثر قیمت اس کے حیطہ کے  \عددیء{1/\sqrt{2}} گنّا ہوتی ہے لہٰذا 
\begin{align}\label{مساوات_مقناطیسی_دور_پیدا_دباو_موثر_قیمت}
E_{rms}=\frac{E_0}{\sqrt{2}}=\frac{2 \pi f N A_c B_0}{\sqrt{2}}=4.44 f N A_c B_0
\end{align}
یہ مساوات بہت اہمیت رکھتی ہے اور ہم اس کو بار بار استعمال کریں گے۔بدلتی برقی دباؤ یا بدلتی برقی رو کی مقدار کی جب بھی ذکر ہو، یہ ان کی مربع کی اوسط کے جزر  یعنی اس کے موثر قیمت  کا ذکر ہوتا ہے۔پاکستان میں گھریلو برقی دباؤ \عددیء{220} وولٹ ہے۔اس کا مطلب ہے کہ اس برقی دباؤ کی موثر قیمت \عددیء{220} وولٹ ہے۔ چونکہ یہ سائن نما ہے لہٰذا اس کی چوٹی \عددی{\sqrt{2} \times 220=311} وولٹ ہے۔
%
\ابتدا{مثال}\شناخت{مثال_مقناطیسی_دور_محرک_برقی_رو_کا_گراف}
شکل میں \عددیء{27} چکر ہیں۔ مرکز کی لمبائی \عددیء{30 } سم جبکہ اس کا رقبہ عمودی تراش \عددیء{229.253} مربع سم ہے۔لچھے  میں گھریلو \عددیء{220} وولٹ موثر برقی دباؤ سے ہیجان  پیدا کیا جاتا ہے۔فہرست کی مدد سے مختلف برقی دباؤ پر محرک برقی رو معلوم کریں اور اس کا گراف بنائیں۔

حل:
	گھریلو برقی دباؤ \عددیء{50} ہرٹز کی سائن نما موج ہوتی ہے یعنی
\begin{align}
v=\sqrt{2} \times 220 \cos (2 \pi  50 t)
\end{align}
مساوات \حوالہ{مساوات_مقناطیسی_دور_پیدا_دباو_موثر_قیمت}  کی مدد سے ہم کثافتِ مقناطیسی بہاؤ کی چوٹی حاصل کرتے ہیں
\begin{align}
B_0=\frac{220}{4.44 \times 50 \times 27 \times 0.0229253}=\SI{1.601}{\tesla}
\end{align}
لہٰذا مرکز میں کثافتِ مقناطیسی بہاؤ صفر سے \عددیء{\mp1.601}  ٹیسلہ کے درمیان تبدیل ہوتی رہتی ہے۔یوں مرکز میں کثافتِ مقناطیسی بہاؤ کی مساوات یہ ہو گی
\begin{align}\label{مساوات_مقناطیسی_دور_سائن_نما_کثافت_بہاو}
B=1.601 \sin \omega t
\end{align}
ہم فہرست کی مدد سے کثافتِ مقناطیسی بہاؤ کے  \عددی{0} سے \عددیء{1.601} ٹیسلہ کے درمیان مختلف قیمتوں پر درکار محرک برقی رو \عددیء{i_{\phi}} معلوم کرنا چاہتے ہیں۔ہم مختلف \عددیء{B} پر فہرست سے مرکز کی \عددیء{H} حاصل کریں گے جو کہ ایک میٹر لمبی مرکز کے لئے درکار ایمپیئر-چکر دیتی ہے۔اس سے \عددیء{30} سم لمبی مرکز کے لئے درکار ایمپیئر-چکر  حل کر کے برقی رو حاصل کریں گے۔

%
\begin{table}
\begin{tabular}{l l l l l | l l l l l}
$i_{\varphi}=\frac{0.3 H}{27}$&$0.3H$&$H$&$B$&$\omega t$&$i_{\varphi}=\frac{0.3 H}{27}$&$0.3H$&$H$&$B$&$\omega t$\\
\hline\\
0.000&0.000&0&0.000&0.000&0.125&3.366&11.22&1.000&0.675\\
0.022&0.600&2&0.040&0.025&0.140&3.777&12.59&1.100&0.757\\
0.033&0.900&3&0.095&0.059&0.166&4.488&14.96&1.200&0.847\\
0.044&1.200&4&0.160&0.100&0.198&5.334&17.78&1.300&0.948\\
0.056&1.500&5&0.240&0.150&0.222&6.000&20&1.340&0.992\\
0.067&1.800&6&0.330&0.208&0.264&7.131&23.77&1.400&1.064\\
0.078&2.100&7&0.440&0.278&0.333&9.000&30&1.480&1.180\\
0.089&2.400&8&0.560&0.357&0.444&12.000&40&1.540&1.294\\
0.100&2.700&9&0.700&0.453&0.556&15.000&50&1.580&1.409\\
0.111&3.000&10&0.835&0.549&0.667&18.000&60&1.601&1.571\\
\hline
\end{tabular}
\caption{محرک برقی رو}
\label{جدول_مقناطیسی_ادوار_محرک_برقی_رو_بالمقابل_کثافت_بہاو}
\end{table}


جدول \حوالہ{جدول_مقناطیسی_ادوار_محرک_برقی_رو_بالمقابل_کثافت_بہاو}  مختلف کثافتِ مقناطیسی بہاؤ کے لئے درکار محرک برقی رو دیتی ہے۔جدول میں  ہر \عددیء{B} کی قیمت پر  \عددیء{\omega t} مساوات \حوالہ{مساوات_مقناطیسی_دور_سائن_نما_کثافت_بہاو}  کی مدد سے حاصل کی گئی ہے۔\عددیء{\omega t} بالمقابل محرک برقی رو کا گراف شکل \حوالہ{شکل_مقناطیسی_ادوار_ہیجان_رو_چال_نظرانداز} میں دیا گیا ہے۔
\انتہا{مثال}
%
\begin{figure}
\centering
\includegraphics{figMagneticCircuitsExcitationCurrentNeglectingHysterisys}
\caption{$M5$ پتری کے مرکز میں $1.6$ ٹیسلہ تک ہیجان پیدا کرنے کے لئے درکار ہیجان انگیز برقی رو۔}
\label{شکل_مقناطیسی_ادوار_ہیجان_رو_چال_نظرانداز}
\end{figure}
برقی لچھے میں برقی دباؤ سے ہیجان پیدا کیا جاتا ہے۔ہیجان شدہ لچھے میں برقی رو کی وجہ سے  مرکز میں مقناطیسی بہاؤ پیدا ہوتا ہے۔ اس برقی رو \عددیء{i_{\phi}} کو ہم ہیجان انگیز برقی رو\فرہنگ{برقی رو!ہیجان انگیز}\حاشیہب{excitation current}\فرہنگ{excitation current}  کہتے ہیں۔

مثال \حوالہ{مثال_مقناطیسی_دور_محرک_برقی_رو_کا_گراف} میں ہیجان انگیز برقی رو معلوم کی گئی جسے شکل \حوالہ{شکل_مقناطیسی_ادوار_ہیجان_رو_چال_نظرانداز} میں دکھایا گیا۔اسے حاصل کرتے وقت مقناطیسی چال\فرہنگ{مقناطیسی چال}\حاشیہب{hysteresis} کو نظر انداز کیا گیا۔شکل \حوالہ{شکل_مقناطیسی_ادوار_ہیجان_رو_بشمول_اثر_چال} میں ہیجان انگیز برقی رو دکھائی گئی ہے جو مقناطیسی چال کو مدِ نظر رکھ کر حاصل کی گئی ہے۔ اس کو سمجھنا نہایت ضروری ہے۔
\begin{figure}
\centering
\includegraphics{figExcitationCurrentFromBHbrillouinCurve}
\caption{ہیجان انگیز برقی رو۔}
\label{شکل_مقناطیسی_ادوار_ہیجان_رو_بشمول_اثر_چال}
\end{figure}
اس شکل  میں دائیں جانب مقناطیسی چال کی خط ہے۔ چونکہ 
\begin{gather}
\begin{aligned}
H l& =N i\\
\varphi&=B A_c
\end{aligned}
\end{gather}
لہٰذا اس خط کو \عددیء{\varphi-i_{\varphi}} کا خط تصور کیا جا سکتا ہے۔شکل کی بائیں جانب مرکز میں سائن نما مقناطیسی بہاؤ \عددیء{\varphi} دکھائی گئی ہے۔یہ سائن نما مقناطیسی بہاؤ کی موج وقت کے ساتھ تبدیل ہوتی ہے۔لمحہ \عددیء{t_1} پر اس موج کی مقدار  \عددیء{\varphi_1} ہو گی۔ یہ شکل میں دکھائی گئی ہے۔اتنی مقناطیسی بہاؤ حاصل کرنے کے لئے درکار ہیجان انگیز برقی رو \عددیء{i_{\varphi1}} مقناطیسی چال کی خط سے حاصل کی جا سکتی ہے۔اس  ہیجان انگیز برقی رو کو شکل میں لمحہ \عددیء{t_1} پر دکھایا گیا ہے۔ 

دھیان رہے کہ اس لمحہ مقناطیسی بہاؤ بڑھ رہی ہے لہٰذا مقناطیسی چال کی خط کا صحیح حصہ استعمال کرنا ضروری ہے۔ شکل \حوالہ{شکل_مقناطیسی_چال}  میں اس حصہ  کو \عددیء{efgb} سے واضح کیا گیا ہے۔

اسی طرح ایک اور لمحہ \عددیء{t_2} جب مقناطیسی بہاؤ کم ہو رہی ہے یہی کچھ دوبارہ شکل  میں ہوتے دکھایا گیا ہے البتہ اس مرتبہ شکل \حوالہ{شکل_مقناطیسی_چال} میں \عددیء{bcde} سے واضح کیا گیا حصہ استعمال کیا گیا ہے۔اس لمحہ پر مقناطیسی بہاؤ \عددیء{\varphi_2} ہے اور اسے حاصل کرنے کے لئے درکار ہیجان انگیز برقی رو \عددیء{i_{\varphi2}} ہے۔

اگر اسی طرح مختلف لمحات پر درکار ہیجان انگیز برقی رو حاصل کی جائے تو ہمیں شکل میں دکھائی گئی  \عددیء{i_{\varphi}} کی خط ملے گی۔یہ ایک غیر سائن نما خط ہے۔

 اگر مرکز میں  \عددیء{B=B_0 \sin \omega t} ہو  تو اِس میں \عددیء{H} اور \عددیء{i_{\varphi}} ایک غیر سائن نما شکل اختیار کر لیتے ہیں۔ اس صورت میں  اِن کے موثر قیمتوں \عددیء{H_{c,rms}} اور  \عددیء{i_{\varphi,rms}} کا تعلق یہ ہے
\begin{align}\label{مساوات_مقناطیسی_دور_دباو_برابر_شدت_ضرب_لمبائی}
N i_{\varphi,rms}=l_c H_{c,rms}
\end{align}
مساوات \حوالہ{مساوات_مقناطیسی_دور_پیدا_دباو_موثر_قیمت}   اور مساوات \حوالہ{مساوات_مقناطیسی_دور_دباو_برابر_شدت_ضرب_لمبائی}  سے ملتا ہے
\begin{align}\label{مساوات_مقناطیسی_دور_درکار_دباو_ضرب_رو}
E_{rms} i_{\varphi,rms}=\sqrt{2} \pi f B_0 H_{c,rms} A_c l_c
\end{align}
یہاں \عددیء{A_c l_c} مرکز کا حجم ہے۔ لہٰذا یہ مساوات ہمیں \عددیء{A_c l_c} حجم کی مرکز  کو \عددیء{B_0} کثافتِ مقناطیسی بہاؤ تک ہیجان کرنے کے لئے درکار \عددیء{E_{rms} i_{\varphi,rms}} بتلاتا ہے۔ ایک مقناطیسی مرکز جس کا حجم  \عددیء{A_c l_c} اور  میکانی کثافت  \عددیء{\rho_c} ہو، اس کی کمیت \عددیء{m_c=\rho_c A_c l_c} ہو گی۔ یوں ہم، ایک کلوگرام  مرکز، کے لئے مساوات \حوالہ{مساوات_مقناطیسی_دور_درکار_دباو_ضرب_رو}   کو یوں لکھ سکتے ہیں
\begin{align}
P_a=\frac{E_{rms} i_{\varphi,rms}}{m_c}=\frac{\sqrt{2} \pi f}{\rho_c} B_0 H_{c,rms}
\end{align}
دیکھا جائے تو کسی ایک تعدد  \عددیء{f} پہ \عددیء{P_a} کی قیمت صرف مرکز اور اس میں \عددیء{B_0} پر منحصر ہے، چونکہ \عددیء{H_{c,rms}} خود \عددیء{B_0} پر منحصر ہے۔ اِسی وجہ سے مرکز بنانے والے، اکائی کمیت کے مرکز میں مختلف \عددیء{B_0} پیدا کرنے کیلئے درکار \عددیء{E_{rms} i_{\varphi,rms}}، کو \عددیء{B_0} اور \عددیء{P_a} کے مابین گراف کی شکل میں دیتے ہیں۔ ایسا ہی ایک گراف شکل میں دکھایا گیا ہے۔\حاشیہط{گراف کہاں ہے}
